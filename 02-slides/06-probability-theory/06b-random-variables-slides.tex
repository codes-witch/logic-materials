\documentclass[fleqn,10pt,serif,xcolor=svgnames,xcolor=table,aspectratio=169,handout]{beamer}
% \includeonlyframes{current}
%========================================
% Packages
%========================================

\usepackage[palatino]{../../99-auxiliary-files/00-mypackBeamer}
\usepackage{../../99-auxiliary-files/00-mycommands}
\usepackage{../../99-auxiliary-files/00-myenvironments-beamer}

\usepackage{tikz-qtree}
\usepackage{array}
\usepackage[absolute,overlay]{textpos}
\usepackage{ulem}

\usepackage{pgfplots}

%========================================
% More Layout (Beamer Special)
%========================================

\DefineNamedColor{named}{mycol}{cmyk}{0.6,0.6,0,0}
% \DefineNamedColor{named}{mygray}{cmyk}{0.05,0.05,0.05,0.05}
% \DefineNamedColor{named}{mygraylight}{cmyk}{0.017,0.017,0.017,0.017}

\definecolor{signal1}{rgb}{0.69, 0.25, 0.21}
\definecolor{signal2}{rgb}{1.0, 0.66, 0.07}
\definecolor{signal3}{rgb}{0.39, 0.58, 0.93}
\definecolor{signal4}{rgb}{0.0, 0.4, 0.0}
\definecolor{firebrick}{rgb}{0.7, 0.13, 0.13}
\definecolor{themecolor}{rgb}{0.3, 0.36, 0.33} % feldgrau
\definecolor{darkgray}{rgb}{0.66, 0.66, 0.66}

% \usetheme[height=7mm]{Rochester}
%\usetheme{Warsaw}


\usecolortheme{dove}

% \useoutertheme[compress,subsection=false]{miniframes}

\usecolortheme[named=themecolor]{structure}

\setbeamercolor{title}{fg=themecolor}

% \setbeamercolor{lower separation line head}{bg=white}

%\setbeamercolor{structure}{fg=Brown}
%\setbeamercolor{normal text}{fg=Brown}
%\setbeamercolor{section in head/foot}{bg=gray!40}
%%\setbeamercolor{lower separation line head}{bg=black!40}
%\setbeamercolor*{frametitle}{fg=Black,bg=gray!40}
%\setbeamercolor*{block body}{fg=Brown,bg=gray!00}
%\setbeamercolor*{block title}{fg=Black,bg=gray!40}


% Switch of shadows of boxes
\setbeamertemplate{blocks}[default]

% Frame numbers in footer
\setbeamertemplate{footline}[frame number]

% See-through preview for uncovered
% \setbeamercovered{transparent}

% Switch off navigation panel at bottom right
\beamertemplatenavigationsymbolsempty

% Change Style for itemize markers
% Options are ball, circle, rectangle and default (=triangle)
\setbeamertemplate{items}[circle]



\setcounter{tocdepth}{1}

% Use bullets in enumerates and TOC
\setbeamertemplate{enumerate item}[circle]

% Set color for enumerate/TOC bullets to white
\setbeamercolor*{item projected}{fg=themecolor,bg=gray!00}

\setbeamercolor*{author}{fg=gray!80}

\setbeamerfont*{block title}{size=\normalsize}
\setbeamerfont*{title}{size=\huge}
\setbeamerfont*{subtitle}{size=\large}

% \newcommand{\mygray}[1]{{\color{gray}{#1}}}
% \newcommand{\mycol}[1]{{\color{mycol}{#1}}}

\newcommand{\mycomment}[1]{\hfill {\mygray{#1}}}
\newcommand{\mycom}[1]{\hfill {\mygray{[#1]}}}

\newcommand{\slideFN}[1]{%
  \begin{textblock*}{\paperwidth}(0pt,1.05\textheight)
    \hfill \footnotesize{\mygray{#1}} \hspace{.5em}
  \end{textblock*}}

\newcommand{\pictureslide}[2][current]{
\usebackgroundtemplate{\includegraphics[width=\paperwidth]{#2}}%
\begin{frame}[label=#1]

\end{frame}
}
% code below makes it possible to turn inclusion of frames
% into 'miniframes' off and on with commands:
% \miniframeson and \miniframesoff
% from: http://tex.stackexchange.com/questions/37127/how-to-remove-some-pages-from-the-navigation-bullets-in-beamer

\makeatletter
\let\beamer@writeslidentry@miniframeson=\beamer@writeslidentry
\def\beamer@writeslidentry@miniframesoff{%
  \expandafter\beamer@ifempty\expandafter{\beamer@framestartpage}{}% does not happen normally
  {%else
    % removed \addtocontents commands
    \clearpage\beamer@notesactions%
  }
}
\newcommand*{\miniframeson}{\let\beamer@writeslidentry=\beamer@writeslidentry@miniframeson}
\newcommand*{\miniframesoff}{\let\beamer@writeslidentry=\beamer@writeslidentry@miniframesoff}
\makeatother

\setbeamertemplate{bibliography item}{}


%========================================
% Commands
%========================================

\newcommand{\mycol}[1]{{\textcolor{mycol}{#1}}}
\renewcommand{\markdef}[1]{\textcolor{themecolor}{\textbf{#1}}}
\newcommand{\mygray}[1]{\textcolor{gray}{#1}}
\definecolor{darkgray}{rgb}{0.66, 0.66, 0.66}

\renewcommand{\slideFN}[1]{%
  \begin{textblock*}{\paperwidth}(0pt,0.95\textheight)
    \hfill \footnotesize{\mygray{#1}} \hspace{.5em}
  \end{textblock*}}

\newcommand{\proplog}{\acro{PropLog}}
\newcommand{\predlog}{\acro{PredLog}}

\newcommand{\mult}{\ensuremath{\cdot}}
\def\checkmark{\tikz\fill[scale=0.4](0,.35) -- (.25,0) -- (1,.7) -- (.25,.15) -- cycle;}

%========================================
% Document
%========================================

\title{Probability theory: random variables}
\subtitle{Methods: Logic, Part 6}

\author{Michael Franke}
\date{}


%--------------------------------------

\begin{document}

% --- Horizontal Space Fix ----

\abovedisplayskip=3pt
\abovedisplayshortskip=3pt

\belowdisplayskip=3pt
\belowdisplayshortskip=3pt

\begin{frame}
  \maketitle
\end{frame}

\begin{frame}

  \pgfplotstableread[row sep=\\,col sep=&]{
    xval & commute & wait \\
    0    & 0.0  & 0.1 \\
    1    & 0.0  & 0.4 \\
    2    & 0.0  & 0.2 \\
    3    & 0.0  & 0.1 \\
    4    & 0.0  & 0.1 \\
    5    & 0.1  & 0.0 \\
    6    & 0.2  & 0.0 \\
    7    & 0.4  & 0.0 \\
    8    & 0.2  & 0.0 \\
    9    & 0.1  & 0.0 \\
    10   & 0.0  & 0.0 \\
    11   & 0.0  & 0.0 \\
    12   & 0.0  & 0.0 \\
  }\data

  \begin{minipage}{0.45\linewidth}
    \centering
    \textcolor{themecolor}{\textbf{Beliefs about commute duration}} \\
    \begin{tikzpicture}
      \begin{axis}[
        ybar,
        bar width=.3cm,
        width=7cm,
        height=4cm,
        % legend style={at={(0.5,1)},
        % anchor=north,legend columns=-1},
        % symbolic x coords={0,1},
        xtick=data,
        % nodes near coords,
        % nodes near coords align={vertical},
        xmin=-0.5,xmax=12.5,
        ymin=-0,ymax=1,
        % title={$P(C = c)$},
        xlabel={$c$}
        ]
        \addplot[fill=signal1] table[x=xval,y=commute]{\data};
      \end{axis}
    \end{tikzpicture}
  \end{minipage}
  \hfill
  \begin{minipage}{0.45\linewidth}
    \centering
    \textcolor{themecolor}{\textbf{Beliefs about waiting duration}} \\
    \begin{tikzpicture}
      \begin{axis}[
        ybar,
        bar width=.3cm,
        width=7cm,
        height=4cm,
        % legend style={at={(0.5,1)},
        % anchor=north,legend columns=-1},
        % symbolic x coords={0,1},
        xtick=data,
        % nodes near coords,
        % nodes near coords align={vertical},
        xmin=-0.5,xmax=12.5,
        ymin=-0,ymax=1,
        % title={$P(W = w)$},
        xlabel={$w$}
        ]
        \addplot[fill=signal2] table[x=xval,y=wait]{\data};
      \end{axis}
    \end{tikzpicture}
  \end{minipage}

  \bigskip \pause

  \begin{center}
    \large
    \textcolor{themecolor}{\textbf{Random variable notation}} \\
    $P(T \le 15) \approx 0.48$\\
    where $ T = {\textcolor{signal1}{C}} + {\textcolor{signal2}{W}} + {\textcolor{signal1}{C}}$
  \end{center}

\end{frame}

\begin{frame}
  \frametitle{Random variable}

  A \markdef{random variable} is a function $X \mycolon \Omega \rightarrow \mathds{R}$

\bigskip
\bigskip

  Notation:
  \begin{itemize}
    \item $P(X = x) = P(\set{\omega \in \Omega \mid X(\omega) = x})$
    \item $P(X \le x) = P(\set{\omega \in \Omega \mid X(\omega) \le x})$
    \item $P(C + W + C  \le 15) = P(\set{\omega \in \Omega \mid C_{1}(\omega) + W(\omega) + C_{2}(\omega) \le 15})$
  \end{itemize}



\end{frame}

\begin{frame}
  \frametitle{Example}

  \begin{itemize}
    \item flip coin with bias $\theta$ twice (independent outcomes)
    \item set of elementary outcomes:
    \begin{align*}
      \Omega_{\text{2flips}} = \set{\tuple{\text{heads}, \text{heads}}, \tuple{\text{heads}, \text{tails}},
      \tuple{\text{tails}, \text{heads}}, \tuple{\text{tails}, \text{tails}}}
    \end{align*}
    \item random variable:
    \begin{align*}
      X_{\text{2flips}}^{\theta}(\tuple{\text{tails}, \text{tails}}) & = 0 & X_{\text{2flips}}^{\theta}(\tuple{\text{heads}, \text{tails}}) & = 1 \\
      X_{\text{2flips}}^{\theta}(\tuple{\text{tails}, \text{heads}}) & = 1 &
      X_{\text{2flips}}^{\theta}(\tuple{\text{heads}, \text{heads}}) & = 2
    \end{align*}
    \item Probabilities:
    \begin{align*}
      P(X_{\text{2flips}}^{\theta} = 0) & = (1-\theta)^{2} \\
      P(X_{\text{2flips}}^{\theta} = 1) & = (1-\theta) \theta \\
      P(X_{\text{2flips}}^{\theta} = 2) & = \theta^{2} \\
    \end{align*}
  \end{itemize}

\end{frame}

\begin{frame}
  \frametitle{Example}

\centering
  \pgfplotstableread[row sep=\\,col sep=&]{
    xval                              & fair & biased \\
    0                                 & 0.25  & 0.0625 \\
    1                                 & 0.5   & 0.375 \\
    2                                 & 0.25  & 0.5625 \\
  }\twoFlips

  \begin{tikzpicture}
    \begin{axis}[
      ybar,
      bar width=.5cm,
      width=7cm,
      height=4cm,
      % legend style={at={(0.5,1)},
      % anchor=north,legend columns=-1},
      % symbolic x coords={0,1},
      xtick=data,
      nodes near coords,
      % nodes near coords align={vertical},
      xmin=-0.5,xmax=2.5,
      ymin=-0,ymax=1,
      title={$P(X_{\text{2flips}}^{0.5} = x)$},
      xlabel={$x$}
      ]
      \addplot[fill=themecolor] table[x=xval,y=fair]{\twoFlips};
    \end{axis}
  \end{tikzpicture}
  \hfill
  \begin{tikzpicture}
    \begin{axis}[
      ybar,
      bar width=.5cm,
      width=7cm,
      height=4cm,
      % legend style={at={(0.5,1)},
      % anchor=north,legend columns=-1},
      % symbolic x coords={0,1},
      xtick=data,
      nodes near coords,
      % nodes near coords align={vertical},
      xmin=-0.5,xmax=2.5,
      ymin=-0,ymax=1,
      title={$P(X_{\text{2flips}}^{0.75} = x)$},
      xlabel={$x$}
      ]
      \addplot[fill=themecolor] table[x=xval,y=biased]{\twoFlips};
    \end{axis}
  \end{tikzpicture}

\end{frame}


\begin{frame}
  \frametitle{Expected value}

\markdef{Expected value of a random variable}
\begin{align*}
  \mathds{E}_X = \sum_{x} x \mult P(X = x) &&& \text{e.g., } \mathds{E}_{T} \approx 15.67
\end{align*}

\bigskip
\bigskip
\bigskip

\markdef{Expected value of a function}
\begin{align*}
  \mathds{E}_X f = \sum_{x} f(x) \mult P(X = x) &&& \text{e.g., with } f(x) = x^{2}, \text{ we have } \mathds{E}_{T} f \approx 250
\end{align*}



\end{frame}

\end{document}
