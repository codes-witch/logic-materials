\PassOptionsToPackage{table}{xcolor}
\documentclass[nobib,nofonts]{tufte-handout}

%\geometry{showframe} % display margins for debugging page layout

%%% MF additions
% \usepackage[table]{xcolor}
\usepackage[nographicx, nohyperref, nosubcaption, nogb4e, nobiblatex]{../99-auxiliary-files/00-mypackages}
\usepackage{../99-auxiliary-files/00-mycommands}
\usepackage{../99-auxiliary-files/00-myenvironments}

\usepackage{titlesec}
\usepackage{etoolbox}
\usepackage{tikz-qtree}
\usepackage{subcaption}

% \titleformat{\section}
% {\large\bfshape}{\thesection}{1em}{}

\setcounter{secnumdepth}{5}
\renewcommand\thesection{\arabic{section}}

% this length controls tha hanging indent for titles
% change the value according to your needs
\newlength\titleindent
\setlength\titleindent{0.7cm}

\pretocmd{\paragraph}{\stepcounter{subsection}}{}{}
\pretocmd{\subparagraph}{\stepcounter{subsubsection}}{}{}

\titleformat{\chapter}[block]
  {\normalfont\huge\bfseries}{}{0pt}{\hspace*{-\titleindent}}

\titleformat{\section}
  {\normalfont\Large\itshape}{\llap{\parbox{\titleindent}{\thesection\hfill}}}{0em}{}

\titleformat{\subsection}
  {\normalfont\itshape}{\llap{\parbox{\titleindent}{\thesubsection\hfill}}}{0em}{}

\titleformat{\subsubsection}
  {\normalfont\normalsize\itshape}{\llap{\parbox{\titleindent}{\thesubsubsection}}}{0em}{}

\titleformat{\paragraph}[runin]
  {\normalfont\normalsize\itshape}{}{-0.7cm}{}[\xspace \ \ \ \ ]

\titleformat{\subparagraph}[runin]
  {\normalfont\normalsize}{\llap{\parbox{\titleindent}{\thesubsubsection\hfill}}}{0em}{}

\titlespacing*{\chapter}{0pt}{0pt}{20pt}
\titlespacing*{\subsubsection}{0pt}{3.25ex plus 1ex minus .2ex}{1.5ex plus .2ex}
\titlespacing*{\paragraph}{0pt}{3.25ex plus 1ex minus .2ex}{0em}
\titlespacing*{\subparagraph}{0pt}{3.25ex plus 1ex minus .2ex}{0em}

\DefineNamedColor{named}{mygray2}{cmyk}{0.55,0.25,0.25,0.25}
\newcommand{\mygray}[1]{\textcolor{mygray2}{#1}}

%%% Tufte style
\usepackage{graphicx} % allow embedded images
  \setkeys{Gin}{width=\linewidth,totalheight=\textheight,keepaspectratio}
  \graphicspath{{graphics/}} % set of paths to search for images

\usepackage{fancyvrb} % extended verbatim environments
  \fvset{fontsize=\normalsize}% default font size for fancy-verbatim environments

% Standardize command font styles and environments
\newcommand{\doccmd}[1]{\texttt{\textbackslash#1}}% command name -- adds backslash automatically
\newcommand{\docopt}[1]{\ensuremath{\langle}\textrm{\textit{#1}}\ensuremath{\rangle}}% optional command argument
\newcommand{\docarg}[1]{\textrm{\textit{#1}}}% (required) command argument
\newcommand{\docenv}[1]{\textsf{#1}}% environment name
\newcommand{\docpkg}[1]{\texttt{#1}}% package name
\newcommand{\doccls}[1]{\texttt{#1}}% document class name
\newcommand{\docclsopt}[1]{\texttt{#1}}% document class option name
\newenvironment{docspec}{\begin{quote}\noindent}{\end{quote}}% command specification environment

\newcommand{\proplog}{\acro{PropLog}}
\newcommand{\EFSQ}{\ensuremath{\mathit{EFSQ}}\xspace}

%%%%%%%%%%%%%%%%%%%%%%%%%%%%%%%%%%%%%%%%%%%%%%%%%%

% \usepackage[sc,osf]{mathpazo}
% \linespread{1.05}



\title{Natural deduction for propositional logic}

\author[M.~Franke]{Michael Franke}

\date{} % without \date command, current date is supplied

\begin{document}

\maketitle

\begin{abstract}
\noindent
Semantic vs. syntactic approach; derivation; derivation rule; soundness \& completeness.
\end{abstract}

\section{Semantic vs.~syntactic approach to logical inference}

A logic usually consists of three major ingredients:
\begin{enumerate}[(i)]
  \item a \emph{syntax} defining the formulas of the language,
  \item a \emph{semantics} that assigns a meaning to each formula, and
  \item an \emph{axiomatic proof system}.
\end{enumerate}
We have already seen (i) and (ii) for \proplog.
Now we are going to look at the final ingredient: an axiomatic proof system.
There are many different kinds of proof systems for \proplog.
The specific one we consider here is called \emph{natural deduction}.

The most central notion of logic is to capture which inferences are correct and which are not.
Given premises $\varphi_{1}, \dots, \varphi_{n}$ we want to know whether conclusion $\psi$ follows.
The notion of logical validity is one way of answering this question.\sidenote{Remember that we write the \emph{argument schema} in question as $\varphi_{1}, \dots, \varphi_{n} / \psi$ and we write  $\varphi_{1}, \dots, \varphi_{n} \models \psi$ if that argument schema is indeed logically valid. Concretely, we said that $\varphi_{1}, \dots, \varphi_{n} \models \psi$ iff any valuation function $V$ that makes the premisses true also makes the conclusion true.}
The definition of logical validity was a semantic one, referring to \emph{valuation functions}, i.e., constructions that are meant to capture the meaning of formulas.

\begin{figure}
  \centering
  \begin{tabular}{clcl}
  1. & $p \vee (q \wedge \neg q)$ & & ass. \\
  2. & $p \rightarrow r$ & & ass. \\
  $\vdots$ & \mygray{[ \dots derivation steps \dots]} & &   \\
  n. & $r$ & &   \\
\end{tabular}
\caption{Sketch of a natural deduction derivation for argument schema $p \vee (q \wedge \neg q)$, $p \rightarrow r$ / $r$.}
\label{fig:deduction-schema}
\end{figure}

We can also take a syntactic approach to delineating what is good logical reasoning and what is not.
In this approach we would characterize the $\varphi_{1}, \dots, \varphi_{n} / \psi$ as an instance of good logical reasoning if and only if there exists a derivation (or proof) which, intuitively speaking, starts with the premisses as assumptions and then leads to $\psi$, following only acceptable rules of inference.
Figure~\ref{fig:deduction-schema} gives a sketch of a natural deduction derivation for argument schema $p \vee (q \wedge \neg q)$, $p \rightarrow r$ / $r$.
Starting with the premisses as assumptions, we use a finite set of derivation steps to end up with the desired conclusion $r$.
Generally speaking, a \emph{derivation} of natural deduction is a finite sequence of formulas such that every formula is either
\begin{enumerate}[(i)]
  \item an assumption,
  \item an additional assumption, or
  \item it is introduced by a legitimate derivation rule\sidenote{We will introduce the set of derivation rules for our system of natural deduction presently. But first we want to concentrate on the conceptual motivation for having such a system in the first place.} based on previous formulas in the sequence.
\end{enumerate}
If a derivation of $\psi$ from premisses $\varphi_{1}, \dots, \varphi_{n}$ exists, we write $\varphi_{1}, \dots, \varphi_{n} \vdash \psi$.

\section{Soundness \& completeness}

Ideally, the semantic definition of ``good inference'' (in terms of valuation functions) and the syntactic definition of it (in terms of derivations) coincide.
In other words, we would really like to have a theorem that tells us that, for any argument schema $\varphi_{1}, \dots, \varphi_{n} / \psi$ it holds that:

\begin{align*}
  \varphi_{1}, \dots, \varphi_{n} \vdash \psi \ \ \ \
  \text{iff} \ \ \ \
  \varphi_{1}, \dots, \varphi_{n} \models \psi
\end{align*}

The left-to-right direction (``if $\vdash$, then $\models$'') is called \emph{soundness} of the derivation system.
Soundness requires that what ever can be syntactically derived (what can be proved in the system) is actually logically valid.
In other words, soundness requires that \emph{only} logically valid conclusions can be derived.
Proving soundness is usually easy: we just need to make sure that each derivation rule is ``correct.''

The right-to-left direction (``if $\models$, then $\vdash$'') is called \emph{completeness} of the derivation system.
Completeness requires that we have enough derivation rules to make sure that we can provide a derivation for all logically valid argument schemes.
In other words, completeness requires that \emph{all} logically valid conclusions can be derived.
Proving completeness is usually much more difficult than proving soundness.

The system of natural deduction introduced in the following is both sound and complete.

\section{Symbiosis of semantic \& syntactic approach}

Why would we want to have two things that do the same thing? ---
Good question.
Several reasons.

First of all, the definitions of $\models$ and $\vdash$ are nicely complementary: one is a universal statement (all valuations \dots), the other is an existential statement (there exists a derivation \dots):

  \begin{center}
    \begin{tabular}{ccc}.
      $\varphi_{1}, \dots, \varphi_{n} \vdash \psi$ &
      iff &
      $\varphi_{1}, \dots, \varphi_{n} \models \psi$ \\
      $\Updownarrow$ & & $\Updownarrow$ \\
      for all valuations \dots && there is a derivation \dots
    \end{tabular}
  \end{center}




\section{Derivation rules of natural deduction}

\subsection{Introduction rule for conjunction $I_{\wedge}$}

We may introduce the conjunction $\varphi \wedge \psi$ whenever both the conjuncts $\varphi$ and $\psi$ are available at previous lines $m_{1}$ and $m_{2}$. It does not matter whether $m_{1}$ occurs before $m_{2}$ or the other way around.\sidenote{We adopt the same convention of omitting the outermost parentheses. Strictly speaking, we should write $(\varphi \wedge \psi)$ in line n. of this derivation.}

\bigskip
\noindent \colorbox{mygray!60}{\centering
  \begin{minipage}[t]{0.35\linewidth}
    \textbf{Conjunction Intro $I_{\wedge}$}
  \end{minipage}
  \begin{minipage}[t]{0.55\linewidth}
    \begin{tabular}{clcl}
      $\vdots$ & $\vdots$              & \\
      m$_{1}$  & $\varphi$             &  \\
      $\vdots$ & $\vdots$              & \\
      m$_{2}$  & $\psi$                & \\
      $\vdots$ & $\vdots$              & \\
      n.       & $\varphi \wedge \psi$ & & $I_{\wedge}$, m$_{1}$, m$_{2}$
    \end{tabular}
  \end{minipage}
}
\bigskip

We can use this rule to show that $p, q, r \vdash (r \wedge p) \wedge q$ like so:

\begin{tabular}{clcl}
  1. & $p$ & & ass. \\
  2. & $q$ & & ass. \\
  3. & $r$ & & ass. \\
  4. & $r \wedge p$ & & $I_{\wedge}$, 3, 1  \\
  5. & $(r \wedge p) \wedge q$ & & $I_{\wedge}$, 4, 2  \\
\end{tabular}


\subsection{Elimination rule for conjunction $E_{\wedge}$}

If we have the conjunction $\varphi \wedge \psi$, we are allowed to also derive each conjunct.\sidenote{It is not necessary to derive both, we can also only derive one of the disjuncts.}

\bigskip
\noindent \colorbox{mygray!60}{\centering
  \begin{minipage}[t]{0.35\linewidth}
    \textbf{Conjunction Elim $E_{\wedge}$}
  \end{minipage}
  \begin{minipage}[t]{0.55\linewidth}
    \begin{tabular}{clcl}
      $\vdots$ & $\vdots$              & \\
      m        & $\varphi \wedge \psi$ &  \\
      $\vdots$ & $\vdots$              & \\
      n$_{1}$  & $\varphi$             & & $E_{\wedge}$, m \\
      n$_{2}$  & $\psi$                & & $E_{\wedge}$, m
    \end{tabular}
  \end{minipage}
}
\bigskip

We can use this new rule to show that $p \wedge q \vdash q \wedge p$ like so:

\begin{tabular}{clcl}
  1. & $p \wedge q$ & & ass. \\
  2. & $p$          & & $E_{\wedge}$, 1 \\
  3. & $q$          & & $E_{\wedge}$, 1 \\
  4. & $q \wedge p$ & & $I_{\wedge}$, 3, 2
\end{tabular}


\subsection{Elimination rule for implication $E_{\rightarrow}$}

If we have $\varphi \rightarrow \psi$ and $\varphi$ somewhere in our derivation (no matter which one comes first), we can derive $\psi$.

\bigskip
\noindent \colorbox{mygray!60}{\centering
  \begin{minipage}[t]{0.35\linewidth}
    \textbf{Implication Elim $E_{\rightarrow}$}
  \end{minipage}
  \begin{minipage}[t]{0.55\linewidth}
    \begin{tabular}{clcl}
      $\vdots$ & $\vdots$                   & \\
      m$_{1}$  & $\varphi \rightarrow \psi$ &  \\
      $\vdots$ & $\vdots$                   & \\
      m$_{2}$  & $\varphi$                  &  \\
      $\vdots$ & $\vdots$                   & \\
      n        & $\psi$                     & & $E_{\rightarrow}$, m$_{1}$, m$_{2}$
    \end{tabular}
  \end{minipage}
}
\bigskip

Using this rule, we can show that $p \wedge r, r \rightarrow q \vdash p \wedge q$:

\begin{tabular}{clcl}
  1. & $p \wedge r$      & & ass. \\
  2. & $r \rightarrow q$ & & ass.  \\
  3. & $p$               & & $E_{\wedge}$, 1  \\
  4. & $r$               & & $E_{\wedge}$, 1  \\
  5. & $q$               & & $E_{\rightarrow}$, 2, 4 \\
  6. & $p \wedge q$      & & $I_{\wedge}$, 3, 5
\end{tabular}

\subsection{Introduction rule for implication $I_{\rightarrow}$}

The introduction rule for implication is slightly more complex.
The idea is this.
We can introduce $\varphi \rightarrow \psi$ if it is possible to derive $\psi$ from the additional assumption that $\varphi$.
We therefore allow \emph{additional, temporary assumptions} to be introduced in order to make ``thought experiments'' like imagining that some formula was given as well.
We use special notation to note where such an additional assumption was made and where this assumption is dropped again.\sidenote{Notice that we do not need to write down which previous lines this rule operates on as this is implicit in the notation used for marking the ``thought experiment'' or better put: the scope of the additional assumption.}
We are \emph{not} allowed to use any of the formulas derived in lines $m$ to $n-1$ after dismissing the additional assumption in line $n$.

\bigskip
\noindent \colorbox{mygray!60}{\centering
  \begin{minipage}[t]{0.35\linewidth}
    \textbf{Implication Elim $E_{\rightarrow}$}
  \end{minipage}
  \begin{minipage}[t]{0.55\linewidth}
    \begin{tabular}{cclcl}
                         & $\vdots$  & $\vdots$                   & \\
      \cline{1-2} \vline & m         & $\varphi$                  & & add. ass.  \\
      \vline             & $ \vdots$ & $\vdots$                   & \\
      \vline             & n-1       & $\psi$                     & & \\ \hline
                         & n         & $\varphi \rightarrow \psi$ & & $I_{\rightarrow}$
    \end{tabular}
  \end{minipage}
}
\bigskip

We can use this rule to show that $\vdash (p \wedge q) \rightarrow q$:

\begin{tabular}{cclcl}
   \cline{1-2} \vline & 1. & $p \wedge q$                        & & ass.  \\
   \vline             & 2. & $q$                                 & & $E_{\wedge}$, 1 \\ \hline
                      & 3. & $\vdash (p \wedge q) \rightarrow q$ & & $I_{\rightarrow}$ \\
\end{tabular}

Another example, with explicit assumptions given is the following derivation showing that  $(p \wedge q) \rightarrow r \vdash (q \wedge p) \rightarrow r$:

\begin{tabular}{cclcl}
                     & 1. & $(p \wedge q) \rightarrow r$ & & ass. \\
  \cline{1-2} \vline & 2. & $q \wedge p$                 & & ass. \\
  \vline             & 3. & $q$                          & & $E_{\wedge}$, 2  \\
  \vline             & 4. & $p$                          & & $E_{\wedge}$, 3  \\
  \vline             & 5. & $p \wedge q$                 & & $I_{\wedge}$, 4, 3  \\
  \vline             & 6. & $r$                          & & $E_{\rightarrow}$, 1, 5  \\ \hline
                     & 7. & $(q \wedge p) \rightarrow r$ & & $I_{\rightarrow}$
\end{tabular}

\subsection{Introduction rule for disjunction $I_{\vee}$}

A disjunction $\varphi \vee \psi$ can be introduced whenever at least one disjunct is available in the derivation.

\bigskip
\noindent \colorbox{mygray!60}{\centering
  \begin{minipage}[t]{0.35\linewidth}
    \textbf{Disjunction Intro $I_{\vee}$}
  \end{minipage}
  \begin{minipage}[t]{0.55\linewidth}
    \begin{tabular}{clcl}
      $\vdots$ & $\vdots$                   & \\
      m & $\varphi$  &  \\
      $\vdots$ & $\vdots$                   & \\
      n$_{1}$ & $\varphi \vee \psi$ & & $I_{\vee}$, m \\
      n$_{2}$ & $\psi \vee \varphi$ & & $I_{\vee}$, m
    \end{tabular}
  \end{minipage}
}
\bigskip

\textcolor{red}{insert example}

\subsection{Eliminiation rule for disjunction $E_{\vee}$}

Intuitively, we can conclude $\chi$ from a disjunction $\varphi \vee \psi$ when $\chi$ follows from $\varphi$ and also follows from $\psi$.

\bigskip
\noindent \colorbox{mygray!60}{\centering
  \begin{minipage}[t]{0.35\linewidth}
    \textbf{Disjunction Elim $E_{\vee}$}
  \end{minipage}
  \begin{minipage}[t]{0.55\linewidth}
    \begin{tabular}{clcl}
            $\vdots$ & $\vdots$                   & \\
      m$_{1}$        & $\varphi \vee \psi$        &  \\
            $\vdots$ & $\vdots$                   & \\
      m$_{2}$        & $\varphi \rightarrow \chi$ &  \\
            $\vdots$ & $\vdots$                   & \\
      m$_{3}$        & $\psi  \rightarrow \chi$   &  \\
            $\vdots$ & $\vdots$                   & \\
      n             & $\chi$                     & & $E_{\vee}$, m$_{1}$, m$_{2}$, m$_{3}$
    \end{tabular}
  \end{minipage}
}
\bigskip

\textcolor{red}{insert example}

\subsection{Elimination rule for negation $E_{\neg}$}

Negation is tricky in natural deduction.
Though we will speak of an elimination rule for negation, strictly speaking we cannot just eliminate negation if we just have a formula $\neg \varphi$.
But we can draw inferences from a negation like $\neg \varphi$ which are ``reductive'' in a sense: if we have derived both $\neg \varphi$ and $\varphi$ we have derived a contradiction, which can be very informative.\sidenote{Remember that the strategy of an indirect proof is to make a certain assumption in order to show that this will result in a contradiction.}
So, the elimination rule for negation can best be thought of as an introduction rule for the sign $\bot$ which we use as a special symbol for a contradiction.\sidenote{Strictly speaking, we should introduce $\bot$ into the language of \proplog as a special formula, which can be used exactly like a proposition letter. We should the say that for all $V$ it's always the case that $V(\bot) = 0$.}

\bigskip
\noindent \colorbox{mygray!60}{\centering
  \begin{minipage}[t]{0.35\linewidth}
    \textbf{Negation Elim $E_{\neg}$}
  \end{minipage}
  \begin{minipage}[t]{0.55\linewidth}
    \begin{tabular}{clcl}
            $\vdots$ & $\vdots$       & \\
      m$_{1}$        & $\neg \varphi$ &  \\
            $\vdots$ & $\vdots$       & \\
      m$_{2}$        & $\varphi$      &  \\
            $\vdots$ & $\vdots$       & \\
      n             & $\bot$         & & $E_{\neg}$, m$_{1}$, m$_{2}$
    \end{tabular}
  \end{minipage}
}
\bigskip


\textcolor{red}{insert example}

\subsection{Introduction rule for negation $I_{\neg}$}

The rule for introducing a negation follows the idea of an indirect proof.
We make an additional assumption that $\varphi$.
If we manage to derive from this assumption a contradiction (denoted as $\bot$), we have derived $\neg \varphi$.\sidenote{Negation introduction is essentially a derivation of $\varphi \rightarrow \bot$ which is logically equivalent to $\neg \varphi$.}

\bigskip
\noindent \colorbox{mygray!60}{\centering
  \begin{minipage}[t]{0.35\linewidth}
    \textbf{Negation Intro $I_{\neg}$}
  \end{minipage}
  \begin{minipage}[t]{0.55\linewidth}
    \begin{tabular}{cclcl}
                         &                   $\vdots$ & $\vdots$       & \\
      \cline{1-2} \vline & m                          & $\varphi$      & & add. ass.  \\
      \vline             &                   $\vdots$ & $\vdots$       & \\
      \vline             & n-1                        & $\bot$         & & \\ \hline
                         & n.                         & $\neg \varphi$ & & $I_{\neg}$
    \end{tabular}
  \end{minipage}
}
\bigskip

\textcolor{red}{insert example}

\subsection{Repetition rule $R$}

We allow to just repeat any previously derived formula.
This is really just for readability of a derivation.

\bigskip
\noindent \colorbox{mygray!60}{\centering
  \begin{minipage}[t]{0.35\linewidth}
    \textbf{Repetition}
  \end{minipage}
  \begin{minipage}[t]{0.55\linewidth}
    \begin{tabular}{clcl}
            $\vdots$ & $\vdots$  & \\
      m              & $\varphi$ &  \\
            $\vdots$ & $\vdots$  & \\
      n              & $\bot$    & & $R$, m
    \end{tabular}
  \end{minipage}
}
\bigskip

\subsection{Ex Falso Sequitor Quodlibet Rule $\EFSQ$}

The derivation rules introduced so far are still not enough to produce a derivation for every valid argument schema.
One example is the logically valid argument schema $p \vee q$, $\neg p$ / $q$.
In order to obtain a system in which $p \vee q, \neg p \vdash q$, we need to introduce another rule, such as the (in)famous \emph{ex falso sequitur quodlibet} (\EFSQ) rule.\sidenote{It is rather difficult to prove that there cannot be a derivation without this (or an equivalent rule). For our purposes, let's just accept that this is so.}
The \EFSQ rule allows for the derivation of \emph{any} formula if a contradiction has been derived.
This is useful, of course, particularly when the contradiction is derived from an additional assumption, as in the introduction of implication (see example below).

\bigskip
\noindent \colorbox{mygray!60}{\centering
  \begin{minipage}[t]{0.35\linewidth}
    \textbf{Repetition}
  \end{minipage}
  \begin{minipage}[t]{0.55\linewidth}
    \begin{tabular}{clcl}
            $\vdots$ & $\vdots$  & \\
      m              & $\bot$ &  \\
            $\vdots$ & $\vdots$  & \\
      n              & $\varphi$    & & $\EFSQ$, m
    \end{tabular}
  \end{minipage}
}
\bigskip

Here is a derivation showing that $p \vee q, \neg p \vdash q$:

\begin{tabular}{cclcl}
                     & 1. & $p \vee   q$                        & & ass.  \\
                     & 1. & $\neg p$                            & & ass.  \\
  \cline{1-2} \vline & 3. & $p$                                 & & add. ass.  \\
  \vline             & 4. & $\bot$                              & & $E_{\neg}$, 2,3 \\
  \vline             & 5. & $q$                                 & & $\EFSQ$, 4\\ \hline
                     & 6. & $p \rightarrow q $ & & $I_{\rightarrow}$ \\
  \cline{1-2} \vline & 7. & $q$                                 & & add. ass.  \\
  \vline             & 8. & $q$                                 & & $R$, 7\\ \hline
                     & 9. & $q \rightarrow q $ & & $I_{\rightarrow}$ \\
                     & 10. & $q$ & & $E_{\vee}, 1,6,9$ \\
\end{tabular}


\subsection{Double-negation elimination rule $E_{\neg\neg}$}

While it may seem innocuous to conclude $\varphi$ from a doubly negated statement like $\neg\neg\varphi$, from the point of view of derivations (think: proofs), this is not so.
In fact, the rules introduced so far do not allow for the elimination of double negation.
We have to introduce a separate rule for this.

\bigskip
\noindent \colorbox{mygray!60}{\centering
  \begin{minipage}[t]{0.35\linewidth}
    \textbf{Double-Neg Elim $E_{\neg\neg}$}
  \end{minipage}
  \begin{minipage}[t]{0.55\linewidth}
    \begin{tabular}{clcl}
            $\vdots$ & $\vdots$  & \\
      m              & $\neg \neg \varphi$ &  \\
            $\vdots$ & $\vdots$  & \\
      n              & $\varphi$    & & $E_{\neg \neg}$, m
    \end{tabular}
  \end{minipage}
}
\bigskip

\textcolor{red}{derive excluded middle}




\end{document}

